% LaTeX file for resume 
% This file uses the resume document class (res.cls)

\documentclass{res} 
\usepackage{multicol}
%\usepackage{helvetica} % uses helvetica postscript font (download helvetica.sty)
%\usepackage{newcent}   % uses new century schoolbook postscript font 
%\setlength{\textheight}{9.5in} % increase text height to fit on 1-page 

\begin{document} 

\name{Jesse Ardonne\\[12pt]}     % the \\[12pt] adds a blank
				        % line after name      

\address{\bf  PRESENT ADDRESS\\Southlake Avenue\\Baton Rouge, LA 70804}
\address{\bf PERMANENT ADDRESS \\Smith Drive\\  Walker , LA 70785}
                                  
\begin{resume}

\section{JOB OBJECTIVE}
	%\vspace{0.005in}
	\rule{\textwidth}{0.5pt}
    To apply my experience in a high-tech growth-oriented organization and utilize strong organizational, communication, and interpersonal relation skills in a challenging and creative environment.      
    
\section{EDUCATION}
	\rule{\textwidth}{0.5pt}
	\vspace{-0.3in} 
	
	{\sl Master of Science}, Electrical Engineering {\bf GPA 3.909}\hfill August 2012 - December 2015\\ 
	Louisiana State University \hfill Baton Rouge, LA\\   
	{\footnotesize Computer Engineering Concentration\\
	PhD Progress: Passed PhD qualifying exam in Fall 2013. Completed all coursework  December 2015.}\\\\
	\vspace{-0.49in}
	\begin{multicols}{2}
	{\footnotesize 
		Research Topics and Interests: \vspace{-0.175in}
		\begin{itemize}
			\item High Performance Computer Architecture
			\item Programming Optimizations
			\item Memory Hierarchy
			\item Cache Management
			\item Parallel Programming Models \ldots
			\end{itemize}
		
		Related Graduate Coursework:  \vspace{-0.175in}
		\begin{itemize}
			\item Computer Algorithms
			\item GPU Programming
			\item Structures of Computers and Computation
			\item Advanced Computer Architecture
			\item Models and Methods for Parallel Computing\ldots
			\end{itemize}
	}
	\end{multicols}

	{\sl Bachelor of Science}, Engineering Technology {\bf GPA 3.817} \hfill August 2008 - May 2012\\
	Southeastern Louisiana University \hfill Hammond, LA\\
	{\footnotesize Dual Concentration in Computer Engineering Technology and Mechanical Engineering Technology}\\\\
		\vspace{-0.49in}
		\begin{multicols}{2}
			{\footnotesize 
				Related Undergraduate Coursework: \vspace{-0.175in}
				\begin{itemize}
					\item Algorithm Design and Implementation
					\item Game Algorithms and Design
					\item Microprocessors and Interfacing \ldots
				\end{itemize} 
				\ %used to balance columns
				\vspace{-0.175in}
				\begin{itemize}
					\item Object-Oriented Programming
					\item Data Structures
					\item Computer Architecture \ldots
				\end{itemize}
			}
		\end{multicols}
		
	
\section{INDUSTRY EXPERIENCE}
	\vspace{0.005in}	
	\rule{\textwidth}{0.5pt}
	\vspace{-0.4in}
	\begin{tabbing}
		\hspace{4.652in}\= \kill % set up two tab positions
		{\bf Engineering Intern} \> \hspace{0.3in}Panama City, FL  \\  
		 Naval Surface Warfare Center PCD \> May 2015 \texttt{-} July 2015	
	\end{tabbing}\vspace{-20pt}      % suppress blank line after tabbing
	\vspace{8pt}The project consisted of autonomously launching and landing a quadcopter on an unmanned surface vehicle. An IRIS quadcopter, equipped with a Pixhawk controller, was outfitted with a Raspberry Pi backseat driver for autonomous control. Python, MAVlink and OpenCV were utilized to interface with the vehicle and provide autonomous control and image processing. Through image processing, an IR light source was identified and used to guide the quadcopter to the landing platform.
		\vspace{-0.2in}
	\begin{tabbing}
		\hspace{4.155in}\=  \kill % set up two tab positions
		{\bf Engineering Intern} \> \hspace{0.815in}Morgan City, LA \\
		Oceaneering \>December 2011 \texttt{-} August 2012
	\end{tabbing}\vspace{-20pt}
	\vspace{8pt}During this internship, my primary objective was maintaining, troubleshooting, and debugging underwater ROV hardware units returning from offshore. Working with two other engineering undergraduates from different engineering backgrounds, additional tasks included creating models in SolidWorks, providing offshore technical support, and wiring equipment modules for deployment.

\section{ACADEMIC EXPERIENCE}
	\vspace{0.005in}	
	\rule{\textwidth}{0.5pt}
	\vspace{-0.4in}
	\begin{tabbing}
		\hspace{4.65in}\= \kill % set up two tab positions
		{\bf Graduate Teaching Assistant} \> \hspace{0.3in}Baton Rouge, LA \\
		Louisiana State University \> August 2015 \texttt{-} Present
	\end{tabbing}\vspace{-20pt}
	\vspace{8pt}Teaching assistant for EE 4755 \texttt{-} Digital Design Using HDLs and EE 3752 \texttt{-} Microprocessor Systems.
	\vspace{-0.35in}
	\begin{tabbing}
		\hspace{4.38in}\=  \kill % set up two tab positions
		{\bf Graduate Research Assistant}\> \hspace{0.62in}Baton Rouge,LA \\
		Louisiana State University     \>August 2012 \texttt{-} March 2015
	\end{tabbing}\vspace{-20pt}      % suppress blank line after tabbing
	\vspace{8pt}Research in Computer Architecture, Hardware Engineering, Cache Management, Transactional Memory, and Memory Hierarchy which included GPU cache bypassing and warp throttling, hardware implementation of transactional memory on GPUs, heterogeneous shared cache resource contention algorithms, and microarchitectural comparisons between Intel Xeon Phi and Nvidia GPU accelerators. Held the Economic Development Assistantship.
	\vspace{-0.2in}
	\begin{tabbing}
		\hspace{4.35in}\= \kill % set up two tab positions
		{\bf Undergraduate Research Assistant} \>\hspace{0.75in}Hammond, LA\\
		Southeastern Louisiana University \>  February 2010 \texttt{-} June 2010
		
	\end{tabbing}\vspace{-20pt}
	\vspace{8pt}Assisted in faculty research projects by testing incoming hardware peripherals via programming and hyperterminal. Presented research on internet-based teleoperations of a mobile robot (Pioneer 3AT).
	
	{\bf MIPS Simulator}\hfill Fall 2012\\
	For my graduate Computer Architecture class, I completed a solo project to program a MIPS simulator in C. The input takes a text file filled with 32 bit binary strings that are MIPS machine instructions and accurately processes the instructions cycle by cycle through an 8-stage pipeline with port forwarding. After each cycle was executed, a snapshot was printed with various data such as program counter, pipeline status, stalled instructions, register values, memory, and branches taken. This project was the foundation to three years of Computer Architecture research using other simulators such as MacSim and GPGPU.\\\\
	{\bf Wii60 Project} \hfill Fall 2011 \texttt{-} Spring 2012 \\
	The original intention was to play the Xbox 360 with a Wiimote, but it quickly turned into a much larger project. I built a circuit that centered around an mBed that directly interfaced with 2 joysticks, 2 triggers, and 16 buttons on a CG Wireless Xbox 360 controller. My final result included two working configurations: Wiimote and keyboard/mouse combination. Hardware inputs for PlayStation 2 and Nintendo 64 were also successfully implemented.
	
\section{TECHNICAL SKILLS}
	\vspace{0.005in}	
	\rule{\textwidth}{0.5pt}	
		\textbullet \ Programming: Most experienced in C / C++. Some experience in Java, JavaScript, Python, Bash.\\
		\textbullet \ Debugging: Experienced with GDB. Have used V-Tune and CUDA Visual Profiler for fine-tuning optimizations.\\
		\textbullet \ Microcontrollers / Embedded Systems: Most experienced with mBed, Arduino, and Raspberry Pi. Some experience with Launchpad, NETduino, and Pixhawk.\\
		\textbullet \ Game Development: Proficient with the Unity game engine. Have also used \textit{Blam!}, Allegro, and Unreal game engines.\\
		\textbullet \ Modeling: Most experienced in 3D Studio Max. Some experience in PSPICE, AutoCAD, and SolidWorks.\\
		\textbullet \ Robotic Platforms: Have worked with IRIS quadcopter and Pioneer 3AT models.

\section{HONORS AND AWARDS}
	\vspace{0.005in}	
	\rule{\textwidth}{0.5pt}
		\begin{multicols}{2}
			{
				\begin{itemize}
					\item Microsoft 2014 LSU College Coding Competition Winner
					\item Economic Development Assistantship
				\end{itemize} 
				\ %used to balance columns
				\vspace{-0.175in}
				\begin{itemize}
					\item Southeastern Honor Scholarship
					\item Air Force Math and Science Award
				\end{itemize}
			}
		\end{multicols}	

\section{PROFESSIONAL MEMBERSHIPS}
	\vspace{0.005in}	
	\rule{\textwidth}{0.5pt}
		\begin{multicols}{2}
			{
				\begin{itemize}
					\item IEEE Member - 6 years
				\end{itemize} 
				\ %used to balance columns
				\vspace{-0.175in}
				\begin{itemize}
					\item Held government security clearance
				\end{itemize}
			}
		\end{multicols}


\section{EXTRACURRICULAR ACTIVITIES}
	\vspace{0.005in}	
	\rule{\textwidth}{0.5pt}
	{\bf Halo Mods}\\
	For a personal project, I wrote an application in C++/CLI capable of modifying Halo related Xbox executables. I also developed a related tool in python for decompressing and recompressing the Halo map files. The application loads a Halo Xbox executable file and allows changing the title, build number, and maps directory, providing a method of installing of multiple Halo copies to a single disc or directory. The Python tool was scripted for quick decompression of the map files using the zlib library to modify the maps and recompress to the correct structure for loading on the Xbox. For a decade, using the official tools released, I modeled numerous custom levels using 3D Studio Max. Next, using a combination of official tools, community driven applications, and hex editing, I further modified the levels to have custom weapons, vehicles, and playable characters.
		
		
	\begin{tabbing}
		\hspace{4.475in}\= \kill % set up two tab positions
		{\bf Aquatic Robotics Camp} \> \hspace{0.475in}Madisonville, LA \\
		Lake Pontchartrain Basin Foundation \> June 2010 and June 2011
	\end{tabbing}\vspace{-20pt}
	\vspace{8pt}Volunteered 80 hours total at the Lake Pontchartrain Basin Maritime Museum for the 2010 and 2011 Aquatic Robotics Program to teach younger students about the fundamentals of underwater Remotely Operated Vehicles.
	
	\begin{tabbing}
		\hspace{3.125in}\= \kill % set up two tab positions
		{\bf Virtual / Physical Judge} \> \hspace{1in}Hammond / Baton Rouge, LA \\
		National Engineers Future City Competition \> January 2013, January 2014, and January 2015
	\end{tabbing}\vspace{-20pt}
	\vspace{8pt}Assessed and evaluated virtual cities submitted by the participating groups using SimCity4 and SimCity5. Also attended the event to provide feedback on the physical model and presentations.


\section{PUBLICATIONS AND PRESENTATIONS}
	\vspace{0.005in}	
	\rule{\textwidth}{0.5pt}
	Shaoming, C., Hu Y., Zhang Y., Peng L., Ardonne, J., Irving, S., Srivastava, A. (2014). \textit{Increasing Off-Chip Bandwidth in Multi-Core Processors with Switchable Pins}. International Symposium of Computer Architecture (ISCA).\\
	
	\vspace{-0.25in}
	Zhang, Y., Irving, S., Peng, L., Fu, X., Koppelman, D., Zhang, W., Ardonne, J. (2015).  \textit{Design Space Exploration for Device and Architectural Heterogeneity in Chip-Multiprocessors}.  Elsevier Microprocessors and Microsystems. \\
	
	\vspace{-0.25in}
	Martinez, C., Ardonne, J., Namira M. (2010). \textit{Internet-Based Teleoperation of a Mobile Robot}. Southeastern Louisiana University Student Showcase.
	    
	   
\section{REFERENCES}
	\vspace{0.005in}	
	\rule{\textwidth}{0.5pt}
	\begin{multicols}{2}
		James R. Perkins\\
		Naval Surface Warfare Center, Panama City Division\\
		110 Vernon Avenue \\
		Panama City, FL, 32407 \\
		(850) 636 – 6451 \\
		james.r.perkins@navy.mil\\
		\\\\\\
		Dr. Cris Koutsougeras, Professor \\
		Department of Computer Science and Industrial Technology \\
		Southeastern Louisiana University \\
		307A Fayard Hall \\
		1205 North Oak Street \\
		Hammond, LA 70742 \\
		(985) 549 – 2189 \\
		ck@selu.edu\\
		
		Dr. Jerry Trahan, Associate Professor \\
		Department of Electrical and Computer Engineering \\
		Louisiana State University Electrical Engineering Building \\
		102 South Campus Drive \\
		Baton Rouge, LA 70803 \\
		(225) 578 – 5243 \\
		jtrahan@lsu.edu
		
		Dr. Patrick McDowell, Associate Professor \\
		Department of Computer Science and Industrial Technology\\
		Southeastern Louisiana University \\
		220 Fayard Hall \\
		1205 North Oak Street \\
		Hammond, LA 70742 \\
		(985) 549 – 5506 \\
		Patrick.McDowell@selu.edu\\
	\end{multicols}
	









\end{resume}
\end{document}